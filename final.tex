\hypertarget{memory-management-unit-mmu}{%
\subsubsection{Memory Management Unit
(MMU)}\label{memory-management-unit-mmu}}

\hypertarget{operating-systems-final-exam}{%
\paragraph{Operating Systems Final
Exam}\label{operating-systems-final-exam}}

\hypertarget{author-gabriel-hofer}{%
\paragraph{Author: Gabriel Hofer}\label{author-gabriel-hofer}}

\hypertarget{date-may-2-2021}{%
\paragraph{Date: May 2, 2021}\label{date-may-2-2021}}

\hypertarget{todo}{%
\paragraph{TODO}\label{todo}}

\begin{enumerate}
\def\labelenumi{\arabic{enumi}.}
\tightlist
\item
  print formatted memory map to stdin
\item
  write tests for code
\item
  make page tables for multiple processes
\item
  make an inverted page table
\end{enumerate}

\hypertarget{memory-management}{%
\paragraph{Memory Management}\label{memory-management}}

As we saw in CoA, memory is central to the operation of a modern
computer system. Memory consists of a large array of bytes, each with
its own address. The CPU fetches instructions from memory according to
the value of the program counter. These instructions may cause
additional loading from and storing to specific memory addresses. We
introduced not only the concept of paging, but also virtual memory. If
you decide to this task you need to inside your dash invoke (you have
the option to write this as a dash-function or a free-standing program
that is forked from inside the dash) a simulation of: 1. Paging, using
an array of e elements simulating the simulated physical memory divided
into f frames of b bytes. Your simulation will run q processes (the user
decides how many processes), needing p pages. A memory map that keeps
track of the frames. Each process then accesses memory locations using:
1. Basic method, each process having their own page table (needs to be
stored somewhere), requesting p pages from the physical memory. 2.
Inverted page table, one page table for the system. 2. Page replacement
algorithm. You decide which algorithm to use, but it needs to be tied to
your paging algorithm. 1. How many pages for each process?

\hypertarget{terminology-used-in-this-document}{%
\paragraph{Terminology used in this
Document}\label{terminology-used-in-this-document}}

\textbf{Page} \textbf{Frame} \textbf{Page number} \textbf{Offset}

\hypertarget{details}{%
\paragraph{Details}\label{details}}

\begin{itemize}
\tightlist
\item
  we will be using a 32-bit address space (uint32\_t)
\end{itemize}

\hypertarget{questions}{%
\subsubsection{Questions}\label{questions}}

\hypertarget{how-large-should-your-simulated-memory-be}{%
\paragraph{How large should your simulated memory
be?}\label{how-large-should-your-simulated-memory-be}}

\hypertarget{how-large-is-the-virtual-memory-of-each-process}{%
\paragraph{How large is the virtual memory of each
process?}\label{how-large-is-the-virtual-memory-of-each-process}}

\hypertarget{how-to-simulate-that-a-process-is-accessing-a-memory-location-how-often-in-what-order}{%
\paragraph{How to simulate that a process is accessing a memory
location? How often? In what
order?}\label{how-to-simulate-that-a-process-is-accessing-a-memory-location-how-often-in-what-order}}

\hypertarget{how-to-keep-track-of-what-memory-locations-are-in-use}{%
\paragraph{How to keep track of what memory locations are in
use?}\label{how-to-keep-track-of-what-memory-locations-are-in-use}}

\hypertarget{what-about-shared-frames}{%
\paragraph{What about shared frames?}\label{what-about-shared-frames}}

\hypertarget{are-there-risk-for-race-conditions-how-will-you-handle-it}{%
\paragraph{Are there risk for race conditions? How will you handle
it?}\label{are-there-risk-for-race-conditions-how-will-you-handle-it}}

\hypertarget{how-to-display-all-information-what-makes-sense-for-the-user-to-see}{%
\paragraph{How to display all information, what makes sense for the user
to
see?}\label{how-to-display-all-information-what-makes-sense-for-the-user-to-see}}

\hypertarget{should-the-user-be-allowed-to-decide-what-information-to-see}{%
\paragraph{Should the user be allowed to decide what information to
see?}\label{should-the-user-be-allowed-to-decide-what-information-to-see}}

\hypertarget{should-the-user-be-allowed-to-freeze-the-simulation-at-any-moment-and-explore-the-current-state}{%
\paragraph{Should the user be allowed to freeze the simulation at any
moment and explore the current
state?}\label{should-the-user-be-allowed-to-freeze-the-simulation-at-any-moment-and-explore-the-current-state}}

\hypertarget{usage}{%
\paragraph{Usage}\label{usage}}

\hypertarget{testing}{%
\paragraph{Testing}\label{testing}}
